\documentclass[Main.tex]{subfiles}
\begin{document}

%\part{aa}

\chapter{Diferencia��o de campos}

\section{A derivada direcional e o gradiente}
O conceito de derivada � uma formaliza��o da id�ia de taxa de varia��o. Quando estamos diante de um campo escalar como a distribui��o de temperaturas em um sala, podemos nos pergutar como varia a temperatura quando nos deslocamos na sala em uma dada dire��o. Ou ainda, se uma abelha se desloca com uma dada trajet�ria dentro desta sala, qual � a derivada da temperatura experimentada pela abelha?

Para responder a estas perguntas, vamos partir das id�ias mais fundamentais do C�lculo Diferencial. Ent�o seja um ponto $(x,y,z)$ no espa�o tridimensional e $f(x,y,z)$ um campo escalar. A taxa de varia��o do valor de $f$ na dire��o do vetor unit�rio $\hat{u}=(u_1,u_2,u_3)$ � denotada 
$\frac{\partial f}{\partial \hat{u}}$ e � dada pelo limite de $h$ para zero do quociente entre a varia��o do valor de $f$ na dire��o $h$, ou seja, $f(x+hu_1,y+hu_2,z+hu_3)-f(x,y,z)$ e $h$:
\begin{equation}
 \frac{\partial f}{\partial \hat{u}}=\lim_{h\to 0} \frac{f(x+hu_1,y+hu_2,z+hu_3)-f(x,y,z)}{h}
\end{equation}
Observe que este � um limite do tipo $\frac{0}{0}$ e, supondo diferenciabilidade de $f$, podemos aplicar a regra de De L'H�pital:
\begin{equation}\label{eqderivadaaux}
 \frac{\partial f}{\partial \hat{u}}= \lim_{h\to 0} \frac{\frac{d}{dh}\left[f(x+hu_1,y+hu_2,z+hu_3)-f(x,y,z)\right]}{\frac{d}{dh}h}=\left.\frac{d}{dh}f(x+hu_1,y+hu_2,z+hu_3)\right|_{h=0}
\end{equation}
Aplicamos, agora a regra da cadeia na derivada obtida:
\begin{equation}\begin{array}{lcl}
 \frac{d}{dh}f(x+hu_1,y+hu_2,z+hu_3)&=& 
 \frac{\partial}{\partial x}f(x+hu_1,y+hu_2,z+hu_3)u_1\\[4pt]
 &+&\frac{\partial}{\partial y}f(x+hu_1,y+hu_2,z+hu_3)u_2\\[4pt]
 &+&\frac{\partial}{\partial z}f(x+hu_1,y+hu_2,z+hu_3)u_3.
\end{array}\end{equation}
Substituindo em (\ref{eqderivadaaux}), temos:
\begin{equation}\label{eqderivada}
 \frac{\partial f}{\partial \hat{u}}=\frac{\partial}{\partial x}f(x,y,z)u_1+\frac{\partial}{\partial y}f(x,y,z)u_2+\frac{\partial}{\partial z}f(x,y,z)u_3.
\end{equation}
Esta � a express�o para a derivada direcional. Observamos agora que esta express�o pode ser reinterpretada como o produto interno entre o vetorformado pelas derivadas parciais de $f$ e o vetor dire��o $\hat{u}$:
\begin{equation}\label{eqderivadagrad}
 \frac{\partial f}{\partial \hat{u}}=\underbrace{\left(\frac{\partial}{\partial x}f(x,y,z)\vct{i}+\frac{\partial}{\partial y}f(x,y,z)\vct{j}+\frac{\partial}{\partial z}f(x,y,z)\vct{k}\right)}_{\nablavec f}\cdot \underbrace{\left(u_1\vct{i}+u_2\vct{j}+u_3\vct{k}\right)}_{\hat{u}}
\end{equation}

\section{O operador $\nablavec$}
\section{O divergente}
\section{O rotacional}
 

\end{document}