\documentclass{book}

\usepackage{bigints}
\usepackage{caption}
 \usepackage{tabu}
\usepackage{ccicons}
\usepackage{subfiles}
\usepackage[latin1]{inputenc}
\usepackage[hmargin=2.5cm,vmargin=2.5cm]{geometry}
%\usepackage[a5paper]{geometry}
\usepackage{amsfonts}
\usepackage{amsmath}
\usepackage[T1]{fontenc}
\usepackage[portuges]{babel}
\usepackage{amsthm}
%\usepackage{graphics}
\usepackage[normalem]{ulem}
\usepackage{pstricks,pst-plot}
\usepackage{graphics}
\usepackage{graphicx}
\usepackage{pst-math}
\usepackage{pst-plot}
\usepackage{pst-3dplot}

\usepackage{pst-solides3d}
\usepackage{wrapfig,float}
%\usepackage{hyperref}
\usepackage{pstricks,pst-plot}

\usepackage{pst-circ}


%\usepackage[answerdelayed,lastexercise]{exercise}
\usepackage[lastexercise]{exercise}

\renewcommand{\ExerciseHeaderTitle}{({\it \ExerciseTitle})}
\renewcommand{\ExerciseHeader}{{\textbf{\large\ExerciseName~\ExerciseHeaderNB\ExerciseHeaderTitle\ExerciseHeaderOrigin\medskip}}}

\renewcommand{\AnswerHeader}{\medskip{\textbf{Resposta do exerc�cio \ExerciseHeaderNB}\smallskip}:~}

\newtheorem{teo}{Teorema}
\newtheorem{lem}{Lema}
\newtheorem{prop}{Proposi\c{c}{\~a}o}
\newtheorem{propr}{Propriedade}
\newtheorem{corol}{Corolario}	
\newtheorem{ex}{Exemplo}	
\newtheorem{prob}{Problema}
\newtheorem{obs}{Observa\c{c}{\~a}o}
\newtheorem{defn}{Defini\c{c}{\~a}o}

\newcommand{\vct}[1]{\vec{#1}}
\renewcommand{\sin}{\operatorname{sen}}
\renewcommand{\sinh}{\operatorname{senh}}

\newcommand{\nablavec}{\vec{\nabla}}

\title{C�lculo Vetorial}
\author{F�bio Azevedo, Esequia Sauter}
\date{\today}

\begin{document}

\maketitle


\chapter*{Licen�a}
Este material est� licenciado por seus autores sob a licen�a Atribui��o-CompartilhaIgual 3.0 N�o Adaptada (CC BY-SA 3.0) %ver \url{https://creativecommons.org/licenses/by-sa/3.0/deed.pt}.

%\ccbysa 3.0





\tableofcontents
%\listoffigures
%\listoftables


\subfile{algvet.tex}
\subfile{curvas.tex}
\subfile{superficies}
%\subfile{Campos_escalares_vetoriais}
\subfile{diferenciacao_de_campos}

\chapter{Integra��o de campos}
\section{Integrais de linha}
\section{Campos conservativos}
\section{Teorema fundamental das Integrais de linha}
\section{Integrais de superf�cie}
\section{Teorema da Diverg�ncia de Gauss}
\section{Teorema de Stokes}
\section{Formula��o int�nseca dos operadores diferenciais}

%\chapter{Deriva��o de campos}
%\section{Gradiente}
%\section{Divergente}
%\subsection{Caso particular do Teorema da Diverg�ncia}
%Provar que
%$$\hbox{div}{\bf{F}}(x,y,z)=\lim_{r\to 0} \int_{}\bf{F}\cdot \bf{dS}$$ 
%\section{Rotacional}
%\section{Teorema da diverg�ncia}
%\section{Teorema de Stokes}



\bibliographystyle{acm}
\bibliography{bib}


\end{document} 